\documentclass[
    DIV=12,
    cleardouble=plain,
    headings=normal,
    pdftex,
    %headexclude,footexclude,
    final
]{scrreprt}

%\usepackage{spreadtab}
\usepackage{xspace}
\usepackage[ngerman]{babel}
\usepackage[utf8]{inputenc}
%\usepackage[T1]{fontenc}
\usepackage[pdftex]{graphicx}
\usepackage[bookmarks]{hyperref}
%\usepackage{scrpage2} %old
\usepackage{scrlayer-scrpage}
\usepackage{longtable}
\usepackage{caption}
%\usepackage{pgfplots} not used?
\usepackage{float}
\usepackage{xcolor}
\usepackage{color tbl}
\usepackage{tabularx}
\usepackage{multirow} % tabelle
\usepackage{listings} % code aus datei einbinden
\usepackage{scrhack}
\usepackage{comment}
\usepackage{hyperref}
% \usepackage{minted} % package for swift
\usepackage{enumerate}

% Swift language definition
\lstdefinelanguage{swift}
{
  morekeywords={
    func,if,then,else,for,in,while,do,switch,case,default,where,break,continue,fallthrough,return,
    typealias,struct,class,enum,protocol,var,func,let,get,set,willSet,didSet,inout,init,deinit,extension,
    subscript,prefix,operator,infix,postfix,precedence,associativity,left,right,none,convenience,dynamic,
    final,lazy,mutating,nonmutating,optional,override,required,static,unowned,safe,weak,internal,
    private,public,is,as,self,unsafe,dynamicType,true,false,nil,Type,Protocol,
  },
  morecomment=[l]{//}, % l is for line comment
  morecomment=[s]{/*}{*/}, % s is for start and end delimiter
  morestring=[b]" % defines that strings are enclosed in double quotes
}

\definecolor{keyword}{HTML}{BA2CA3}
\definecolor{string}{HTML}{D12F1B}
\definecolor{comment}{HTML}{008400}

\lstset{
  language=swift,
  basicstyle=\ttfamily,
  showstringspaces=false, % lets spaces in strings appear as real spaces
  columns=fixed,
  keepspaces=true,
  keywordstyle=\color{keyword},
  stringstyle=\color{string},
  commentstyle=\color{comment},
}

% word wrap with a red arrow
\lstset{
  basicstyle=\ttfamily,
  columns=fullflexible,
  frame=single,
  breaklines=true,
  postbreak=\mbox{\textcolor{red}{$\hookrightarrow$}\space},
}

\graphicspath{{./}{./images/}}

% #################################################################

%\begin{document}

\hyphenation{Cha-otn-gsch-werl}
\setlength\headheight{1.75cm}

%\ihead*{\small{Fortgeschrittene Programmierung unter SWIFT (FWPM)}}
\ohead*{\includegraphics[height=0.05\textheight]{fh_logo}}
\pagestyle{scrheadings}

\begin{document}


\ohead*{\includegraphics[height=0.05\textheight]{fh_logo}}
\ifoot*{\small{The Teachify Project}}
\ofoot*{\small{Prof. Dr. Sven Rill}}



\setcounter{secnumdepth}{5}
\setcounter{tocdepth}{5}
\renewcommand{\arraystretch}{1}

\parskip0.5\baselineskip plus 0.125\baselineskip minus 0.25\baselineskip
\parindent0em

\automark*[section]{chapter}

\titlehead{\begin{center}\includegraphics[width=5cm]{fh_logo}\end{center}}
 \title{
  Teachify - Lernspiele für Kinder \\[1em]
  Dokumentation, Spezifikation, Konstruktion
}


\author{Angelina Scheler, Bastian Kusserow, Christian Pfeiffer,\\ Christian Pöhlmann, \href{mailto:jofranz90@gmail.com?subject=Swift-Studienarbeit}{Johannes Franz}, Marcel Hagmann, Maximilian Sonntag,\\Normen Krug, Patrick Niepel,
Phillipp Dümleim, Carl Phillipp Knoblauch}

\date{09.07.2018\\ Vorgelegt bei Prof. Dr. Sven Rill} %##Abgabedatum einfügen


\maketitle
\pagenumbering{roman}
\tableofcontents

%\listoftables

\newpage
\pagenumbering{arabic}

%hier die einzelnen Punkte einfügen
\chapter{Pflichtenheft}
Christian Pfeiffer, Normen Krug \& \href{mailto:jofranz90@gmail.com?subject=Swift-Studienarbeit}{Johannes Franz}


\section{Zielbestimmung}
Es ist eine Lernspiel Software für Grundschulschüler zu entwickeln, welche auf iPads ab iOS Version 10 lauffähig ist. Schüler sollen Lernspiele bzw. Aufgaben bearbeiten können, welche von den Lehrern vorher generiert werden. Nach den Spielen können Schüler ihre eigenen Leistungen ansehen. Auch Lehrer sollen einen Überblick über die Leistungen der eigenen Schüler haben. Die Ausarbeitung der App ist auf ein Semester beschränkt, das bedeutet es stehen 3 Monate zur Umsetzung zur Verfügung. Kurz vor der Abgabe ist die Funktionsfähigkeit der Software durch Tests zu bestätigen.

\subsection{Musskriterien}

\subsubsection{Allgemein}
\begin{enumerate}[a)]
\item Die App muss auf iPads mit iOS 11 laufen
\end{enumerate}



\subsubsection{Login}
\begin{enumerate}[a)]
\item Login für den Schüler
\item Login für den Lehrer
\end{enumerate}

\subsubsection{Schüler}
\begin{enumerate}[a)]
\item Übersicht über alle freigegebenen Spiele (Spiele von der Schnittstelle abrufen)
\item Übersicht über erreichte Punktzahlen
\item Spielbeschreibung anzeigen
\item Spiel spielen
\item Ergebnisse anzeigen
\item Ergebnisse über die Schnittstelle hochladen
\end{enumerate}

\subsubsection{Lehrer}
\begin{enumerate}[a)]
\item Übersicht über alle registrierten Schüler
\item Einladen von Schülern in “Klassen”
\item Anzeigen von Ergebnissen einzelner Schüler
\item Aufgaben erstellen
\end{enumerate}


\subsubsection{Schnittstelle}
\begin{enumerate}[a)]
\item Abrufen der freigegebenen Spiele für einen Schüler
\item Schüler zu Aufgaben einladen (durch den Lehrer)
\item Abspeichern der Ergebnisse der gelösten Aufgaben
\item Abrufen der Ergebnisse der gelösten Aufgaben (Lehrer)
\end{enumerate}


\subsection{Wunschkriterien}
\subsubsection{Allgemein}
\begin{enumerate}[a)]
\item Die App kann auch auf anderen iOS Geräten (iPhone) laufen
\end{enumerate}

\subsubsection{Login}
\begin{enumerate}[a)]
\item Alternative Login Methode für den Lehrer
\end{enumerate}

\subsubsection{Schüler}
\begin{enumerate}[a)]
\item Übersicht über die vergangen Spiele
\item Lösen von Aufgaben unter Zeitdruck
\item Kindgerechte und einfache Menügestaltung
\item Schüler soll es ermöglicht werden in eigenem Tempo zulernen
\end{enumerate}

\subsubsection{Lehrer}
\begin{enumerate}[a)]
\item Einteilung der Schüler in Klassen
\item Ranking der Spielergebnisse der Klassen
\item Individuelle Förderung von Schülern
\end{enumerate}


\subsubsection{Schnittstelle}
\begin{enumerate}[a)]
\item Detailliertes Speichern der Spiele für eine Auswertung (gebrauchte Antwortzeit, Statistiken für Klassen)
\item Abspeichern von Bildern
\end{enumerate}


\subsection{Abgrenzungskriterien}
Das System besitzt keine Schnittstellen zu anderen Produkten.
Es existiert keine automatische Erfassung von Benutzern aus Fremddaten.

\section{Produkteinsatz}
Die App soll als Referenz für die Lehre der Fortgeschrittenen Swift 4 Entwicklung des Mobile Computing Studiengangs an der Hochschule Hof dienen.\\
\\
Hypothetisch soll die App im Rahmen des Grundschulunterrichts eingesetzt werden. Hierbei hat jeder Schüler ein eigenes Tablet und kann mit diesem Aufgaben bearbeiten. Der Lehrer kann den Schülern Aufgaben zuweisen und die Aufgabenergebnisse einsehen.  

\subsection{Anwendungsbereiche}
Primär Grundschulen, später Erweiterung für höhere Bildungseinrichtungen denkbar. 

\subsection{Zielgruppen}
Schüler / Studenten und Lehrer bzw. Lehrbeauftragte

\subsection{Betriebsbedingungen}
Um die App in allen Funktionen nutzen zu können, wird ein iPad mit iOS 10 mit Internetverbindung vorausgesetzt. Beim Einloggen als Schüler müssen alle freigeschalteten und geteilten Aufgaben abgerufen werden. Eingeloggte Lehrer sollen eine Übersicht über die verfügbaren Aufgabentypen sowie Schüler bzw. deren Klassen haben. Die Pflege der Schnittstelle soll wartungsfrei sein. Die administrative Gewalt (Zuweisung der Aufgaben, sowie Einblick in die Statistik der Schüler) soll bei den Lehrern stehen.



\section{Produktfunktionen}
\subsection{/F0010/ Einloggen}
Ein beliebiger App Nutzer kann sich über den Startscreen einloggen. 
Als Kennung wird sein iCloud Account verwendet. Die App muss zwischen Schülerbenutzern und Lehrerbenutzern unterscheiden können. Die Unterscheidung dieser zwei Benutzergruppen geschieht durch unterschiedliche Loginverfahren (Zum Login eines Lehrers muss ein Button gedrückt werden,  bzw. ein QR Code verwendet werden).Nach dem Login wird der Benutzer in das seiner Benutzergruppe zugehörige Hauptmenü weitergeleitet (Schüler bzw. Lehrer).


\subsection{/F0020/ Verfügbare Spiele herunterladen (Schüler/Schnittstelle)}
Schülerbenutzer kann im Schülerhauptmenü seine ihm freigegebenen Aufgaben bzw. Spiele einsehen. Das Herunterladen dieser geschieht automatisch nach dem Login. Falls dem Schüler keine Aufgaben freigegeben wurden, wird ihm anstatt der Aufgaben ein Hinweis angezeigt.


\subsection{/F0030/ Spielinformationen anzeigen (Schüler)}
Wählt der Schüler ein ihm freigegebenes Spiel in dem Schülerhauptmenü aus, so werden ihm Informationen über das ausgewählte Spiel (Spielregeln, Schwierigkeit… etc. angezeigt).\\
Über einen ''Spiel starten'' Button, kann der Schüler das Spiel starten. Mit einer Geste bzw. einem Zurückbutton kann er zurück in das Hauptmenü gelangen.


\subsection{/F0040/ Spiel spielen (Schüler)}
Wählt der Schüler den “Spiel starten” Button in den Spielinformationen aus, wird das Spiel gestartet und  er kann die Aufgaben bearbeiten. Nach dem Bearbeiten der Aufgaben werden die Spielergebnisse über die Schnittstelle wieder hochgeladen.



\subsection{/F0050/ Leistungen anzeigen (Schüler)}
Wählt ein Schüler den ''Erfolge Tab'' in dem Schülerhauptmenü aus, so kann er seine Spielerfolge einsehen. Hierbei wird ihm eine Punktzahl angezeigt, welche die Summe der innerhalb der Lernspiele erreichten Punkte ist. Zudem kann er sich eine Übersicht über seine letzten Spiele und die darin richtig bzw. falsch gelösten Aufgaben anzeigen lassen.

\subsection{/F0060/ Schüler in Klassen einladen (Lehrer/Schnittstelle/Schüler)}
Bevor der Lehrer Aufgaben verteilen kann, muss er seine Schüler in eine Klasse einladen. Wenn die Schüler die Einladung der Lehrer annehmen, werden sie der Klasse hinzugefügt.

\subsection{/F0070/ Aufgaben Klassen zuteilen}
Ein Lehrer kann an einzelne Schüler spezielle Aufgaben stellen.

\subsection{/F0080/ Ergebnisse einzelner Schüler anzeigen}
Für den Lehrer muss es möglich sein, die Ergebnisse der Schüler einzusehen.

\chapter{Schülerteil}
Christian Pfeiffer, Normen Krug \& \href{mailto:jofranz90@gmail.com?subject=Swift-Studienarbeit}{Johannes Franz}



\section{Einleitung}
In diesem Abschnitt werden die Ziele und die Motivation des Projektes definiert. Dabei werden unter anderem die Erwartungen an das Projekt genannt.
\subsection{Ziele}
Als Ziele der Studienarbeit wurden folgende Punkte definiert: 
\begin{itemize}
\item Kinderfreundliches Design und Layout
\item Erstellen eines Mathelernspieles 
\item Aufgaben die von Lehrern erstellt werden anzeigen und in ein spielbare Form überführen
\item Die von Schüler beantworteten an den Lehrer weiterleiten
\item Den Schülern die Möglichkeit bieten die Spiele im Endlos Modus, unabhängig der von Lehrer zugewiesenen Aufgaben, zu spielen
\item Lernen eines neuen Apple Frameworks (\textbf{SpriteKit})
\item Erfahrung sammeln in Zusammenarbeit mit anderen Teams
\end{itemize}
\subsection{Motivation}
Die Hauptmotitvation des Projektes war das Lernen und Einarbeiten in neue Apple Frameworks und Erfahrungen sammeln in der Zusammenarbeit mit anderen Teams. Durch den Aufbau der  Studienarbeit war es zwingend notwendig, sich mit anderen Teams zu verständigen und auszutauschen.  
\section{Spezifikation}
% Herausfoderungen im SpriteKit
% Verbindung von SK und der restlichen App
% Refactoring und performance
Als Strategie für die Umsetzung des Projektes wurde das Prinzip "Funktionalität vor UI-Design" gewählt.
\subsection{Mathe Piano Spiel}
Beim Spiel war es wichtig, möglichst schnell einen funktionsfähigen Prototypen zu entwickeln. Dieser wurde im Verlauf des Projektes nach und nach immer weiter verbessert.

% Anbindung des Spiels an einen Zahlen/Aufgabengenerator

% Anbindung des Spiels an die Schnittstelle, die im weiteren Projekt


\subsection{User Interface}

\section{Umsetzung}
Die Umsetzung 
\subsection{Mathe Piano Spiel}


\subsection{User Interface}


\section{Fazit}
Erstes großes Projekt mit vielen Contributern und damit verbundenen Herausforderungen.\\
Umgang mit Git im großen Team.
In Konflikt getretene Herangehensweisen Design vs. Funktion und umgekehrt.
Erfahrung sammeln in der Zusammenarbeit mit anderen Teams %Verweis zu oben

\chapter{Schüler Game Teachbird}
Christian Pöhlamnn

\section{Grundlage}
Grundlegende Information.
\chapter{Das Spiel FeedMe}
Angelina Scheler

\section{Der Entwurf}

Die Idee von FeedMe war es Multiple Choice Fragen interessanter zu gestalten, damit die Schüler motiviert bleiben. Eigenschaften des Spiels:

\begin{itemize}
\item Multiple Choice
\item für alle Fächer geeignet
\item Monster muss mit richtiger Antwort gefüttert
\item bei falscher Antwort wird ein Leben abgezogen
\item verliert der Spieler zu viele Leben, wird das Monster wütend
\end{itemize}

\begin{figure}[H]
	\centering
  \frame{ 
  \includegraphics[width=0.99\textwidth]{images/FeedMeEntwurf.jpg}
  }
	\caption{Erster Entwurf}
	\label{Erster Entwurf}
\end{figure}



\section{Die Umsetzung und Implementierung}

Das Spiel wurde mit dem SpriteKit umgesetzt
\begin{figure}[H]
	\centering
  \frame{ 
  \includegraphics[width=0.6\textwidth]{images/FeedMeScreens.png}
  }
	\caption{Game Screens}
	\label{Game Screens}
\end{figure}


In einem SpriteKit Scene File wurden Color Sprites und Labels als Nodes eingefügt, den Sprites Assets zugewiesen, Skalierung und Position angepasst. (GameFeedMeScene.sks)
In \textit{GameFeedMeSwift.swift} wurden dann die Funktion umgesetzt.

\subsection{Die Nodes}
Für alle benötigten Nodes aus der Scene wurden Variablen angelegt z.B.

		\textit{var frage: SKLabelNode?}

und in der didMove() Methode initialisiert z. B.

		\textit{self.frage = question?.childNode(withName: "frage") as? SKLabelNode}

\subsection{Der Sound}
Über ein SKAudioNode wird der Sound initialisiert. Hintergrundmusik, Touch, falsche/richtige Antwort, Game Over und New Game.
Beispielsweise die Hintergrundmusik als Loop in der didMove() Methode

\textit{let backgroundMusic = SKAudioNode(fileNamed: "sky-loop.wav")
        backgroundMusic.autoplayLooped = true
        addChild(backgroundMusic)}

und beim Bewegen der „Früchte“ als einmaliger Ton in touchesBegan()

\textit{run(SKAction.playSoundFileNamed("beeps.wav", waitForCompletion: false))}

\subsection{Die Touch-Events}
Um die Antworten bewegen zu können sind 3 Funktionen nötig:

\begin{itemize}
\item touchesBegan()
\item touchesMoved()
\item touchesEnded()
\end{itemize}

In der touchesBegan() Funktion wird eine location variable initialisiert, welche die Postion des „touch“ enthält. 
Befindet sich die location auf einer der Antworten wird einer Hilfsvariable „movableNode“ die Postion zugewiesen und die ursprüngliche Position in einer weiteren Hilfsvariable „originalposition“ zwischengespeichert und in der touchesMoved() Funktion ständig aktualisiert. 
Endet das Touch Event wird in der touchesEnded() Methode das „movableNode“ an die „originalPosition“ verwiesen und „movablenode“ zurückgesetzt.

\begin{figure}[H]
	\centering
  \frame{ 
  \includegraphics[scale=0.7]{images/FeedMeTouch.png}
  }
	\caption{Die Touch-Events}
	\label{Die Touch-Events}
\end{figure}




\subsection{Die Animation}

Der Drache wurde über eine SKAction animiert. Zunächst wurde ein Images.Atlas Ordner mit verschiedenen Bildern des Drachens angelegt.

\begin{figure}[H]
	\centering
  \frame{ 
  \includegraphics[scale=0.7]{images/FeedMeDragons.png}
  }
	\caption{Drachen Assets}
	\label{Drachen Assets}
\end{figure}

Die Bilder wurden in ein SKTexture Array „dragonFrames“ bzw. „evilDragonFrames in der buildDragon() Funktion eingelesen. 
In der animateDragon() Funktion wird über dragon.run auf das Array zugegriffen und über ein Zeitintervall die Frames des SKSpriteNode des Drachen geändert.

	\textit{dragon?.run(SKAction.repeatForever(
                	SKAction.animate(with: dragonFrames,
                                 	timePerFrame: 0.5,
                                	 resize: false,
                                	 restore: true)),
                       	 withKey:“walkingInPlaceDragon")}
                        
                        

Die Leben in Form von Herzen(siehe. Abb. GameScreen) werden ebenfalls über eine SKAction animiert.

	einblenden über „fadeIn()“
	\textit{heart?.run(SKAction.fadeIn(withDuration: 2.0))}
	
	ausblenden über „fadeOut()“
	\textit{heart?.run(SKAction.fadeOut(withDuration: 1.0))}
	
	mit „withDuration“, kann die Geschwindigkeit angegeben werden
	
	

\subsection{Das Game Play}

Der User startet mit 3 Leben welche durch die Herzen angezeigt wird, durch ziehen der „Früchte“ auf den Drachen wird die Antwort geprüft, bei falscher Antwort verliert der User ein Leben, befindet er sich kurz vor dem „Game Over“ ändert der Drache die Farbe „wird böse“.

Durch initialisieren eines Integer Arrays welches 4 zufällige Zahlen enthält, werden die Frage, die Antworten generiert und die Antworten den Nodes zufällig zugeordnet.

Ist der User „Game Over“ bzw. hat alle Leben verloren, kann er entweder das Spiel Neustarten oder dieses Beenden über den jeweiligen Button.



\chapter{Lehrer UI, Anbindung an die Schnittstelle, Erstellen und Teilen von Aufgaben}
Philipp Dümlein, Bastian Kusserow, Maximilian Sonntag


\section{Lehrer UI}
Zu Beginn des Projekts musste zunächst eine funktionale und optisch ordentliche Benutzeroberfläche für den Lehrer entwickelt werden. Hierbei waren Ziele moderne UI-Elemente zu verwenden, sowie erste Erfahrungen bei der Implementierung von Views im Code zu sammeln. Ganz besonders zeigt sich dies auf dem Homescreen des Lehrers.
\subsection{Homescreen}
Auf dem Homescreen wurde durch die Kombination von mehreren CollectionViews (CVs), die unterschiedlich angepasst wurden, ein funktionales und übersichtliches UI geschaffen. Die Aufteilung wurde wie folgt implementiert:
\begin{itemize}
\item CV zur Anzeige, Auswahl und Erstellung von Klassen
\item CV zur Anzeige, Auswahl und Erstellung des Faches    
\item Animierter Indikator zur Anzeige des aktuell gewählten Fachs
\item CV zur Anzeige, Auswahl und Erstellung von Dokumenten
\end{itemize}
Eine Besonderheit der CV der Dokumente ist die Sharing-Funktion, welche durch langen Druck auf ein Dokument aufgerufen werden kann. Nach erfolgreichem Sharing wird in einem Overlay der QR-Code mit dem Freigabelink angezeigt. Dieser kann dann von Schülern verwendet werden um Zugriff auf das geteilte Dokument zu bekommen.

\subsection{Erstellen von Aufgaben}
Bei der Erstellung von Aufgaben muss der Lehrer zunächst auswählen um welches Fach es sich handelt. Daraufhin muss ausgewählt werden, welche Art von Aufgaben erstellt werden sollen. Hierbei muss neben dem Fach auch eine genauere Angabe zum Aufgabentyp gemacht werden. Aktuell gibt es Folgende Auswahlmöglichkeiten:
\begin{itemize}
\item Englisch - Vokabeln
\item Englisch - Grammatik
\item Englisch - Synonyme
\item Mathematik - Addition
\item Mathematik - Subtraktion
\item Mathematik - Division
\item Mathematik - Multiplikation
\end{itemize}
Danach muss noch eine Auswahl getroffen werden zu welchem Spiel die Aufgaben auf Schülerseite werden sollen.
Je nach Auswahl des Aufgabentyps werden unterschiedliche Views geöffnet, auf denen spezifische Einstellungen zur Auswahl gesetzt werden.
Im Rahmen der Studienarbeit wurde bisher die Erstellung von Englisch Vokabeltests und von Mathematik-Aufgaben mit allen Grundrechenarten implementiert.


\section{Implementierung}
\subsection{Homescreen}

\subsubsection{Klassen- und Fächer-CVs}
Die CVs zur Anzeige und Auswahl der Klassen und Fächer verfügen neben den einzelnen Elementen auch Einträge zur Anzeige von allen Unterelementen und zum Hinzufügen von Klassen/Fächern.
Letzteres geschieht über einen eigens geschriebenen ViewController, welcher modal als Overlay angezeigt wird. Hier kann der Name für die Klasse bzw. das Fach eingetragen werden. Wird der Save-Button betätigt, wird automatisch ein Objekt in die Cloud hochgeladen.\\Die einzelnen Elemente der CVs können selektiert werden. Daraufhin werden die jeweils darunterliegenden CVs entsprechend angepasst. Als visuelles Feedback für den Nutzer wird bei Klassen der Name des ausgewählten Eintrags fett geschrieben. Für die Fächer gibt es einen weißen Indikator, welcher unterhalb des ausgewählten Eintrags angezeigt wird und beim Wechsel animiert zum neuen Eintrag springt.

\subsubsection{Dokumente-CV}
Die CV zeigt immer die Dokumente des aktuell ausgewählten Fachs einer Klasse. Die wohl wichtigste Funktion ist das Sharing einzelner Dokumente. Dies wurde durch eine Long-Tap-Action und einem sog. \textit{MenuItem} umgesetzt. Wird auf dieses gedrückt, öffnet sich ein modaler SharingController, in dem alle Sharing-Optionen angezeigt werden.\\Hier wird auch der eigene Share-Service \textit{'Sharing'} angezeigt, welcher dafür sorgt, dass die Aufgabe in die Cloud geladen wird und der Link für die Freigabe zurückgeliefert und als QR-Code angezeigt wird. Dieser kann dann von Schülern verwendet werden um Zugriff zur geteilten Aufgabe zu erhalten.

\subsection{Herunterladen der Clouddaten}
Beim Start der Lehrerseite werden zunächst alle bereits durch die App hochgeladenen Daten auf dem iCloud Account des Nutzers heruntergeladen. Aufgrund des Datenmappings ist hierbei die Reihenfolge enorm wichtig. Zur Erklärung:
\begin{itemize}
\item Eine Klasse ist eigenständig
\item Ein Fach gehört zu einer Klasse
\item Ein Dokument gehört zu einem Fach
\item Eine Aufgabe gehört zu einem Dokument
\end{itemize}
Da die jeweiligen Unterobjekte ohne deren Überobjekt nicht erreichbar sind, müssen also zunächst Klassen, dann deren Fächer, daraufhin die jeweiligen Dokumente und zuletzt die entsprechenden Aufgaben heruntergeladen werden.
Für den Download selbst bietet die Schnittstelle zwar entsprechende Methoden, dennoch stellt die Reihenfolge der Downloads ein Problem dar.

\subsubsection{Problematik}
Beim Herunterladen der Daten ist es durch das Mapping zwingend notwendig eine bestimmte Reihenfolge einzuhalten. Da Downloads aber standardmäßig asynchron ablaufen, kann nie genau gesagt werden, wann welche Daten verfügbar sind.
Eine Möglichkeit des Downloads ist die Verschachtelung der einzelnen Aufrufe. Das bedeutet, das zunächst alle Klassen heruntergeladen werden, danach alle Fächer, alle Dokumente und alle Aufgaben. Dies ist zwangsläufig notwendig, im Code allerdings extrem unübersichtlich und unsauber. Daher wurde eine zwar etwas aufwändigere, aber deutlich schönere Lösung implementiert.

\subsubsection{Lösung}
Um die Verschachtelung sauber zu Implementieren wurde auf \textit{Operations} zurückgegriffen. Es wurde eine \textit{BaseOperation} implementiert, welche von \textit{Operation} erbt. Hierdurch kann der genaue Zustand eines Downloads gesteuert/abgerufen werden.
Es wurden für den Download von Klassen, Fächern, Dokumenten und Aufgaben eigene Operationen implementiert, welche von der \textit{BaseOperation} erben.
Beim Download werden nun lediglich die Completionblocks angegeben, um die heruntergeladenen Daten entsprechend zu setzen. 
\\Die einzelnen Operationen werden dann in einer \textit{OperationQueue} ausgeführt. Die Reihenfolge der Downloads wird durch sog. \textit{Dependencies} gesteuert. Die Dependencies sorgen dafür, dass ein in der Reihe später stehender Block erst ausgeführt wird, wenn der vorhergehende beendet ist. Außerdem wird maximal eine Operation gleichzeitig ausgeführt, wodurch zusätzlich verhindert wird, dass Downloads gestartet werden, für die die erforderlichen Daten noch nicht vorhanden sind.

\section{Aufgaben erstellen}
Beim Erstellen von Aufgaben sind insbesondere die Mathematik-Aufgaben interessant. Es kann mit Hilfe zweier Picker ausgewählt werden, in welchem Bereich sich die Operanden der Rechenoperation befinden sollen. Zudem kann angegeben werden, ob diese Zahlen auch negative Werte annehmen dürfen. In Abhängigkeit der Auswahl werden die Picker entsprechend angepasst. Das bedeutet, negative Zahlen sind verfügbar bzw. nicht verfügbar und der Zahlenbereich umfasst immer mindestens eins. Das bedeutet auch, dass beim Verändern der Picker die jeweils andere Seite entsprechend geändert wird, sodass ein positiver Zahlenbereich entsteht.

\section{Fazit}
Der aktuelle Implementierungsstand umfasst den größten Teil der Grundfunkionalität. Das Erstellen und Teilen von Aufgaben ist grundsätzlich möglich und funktioniert. Im Rahmen der Studienarbeit konnten aus zeitlichen Gründen nicht mehr Aufgabentypen, Fächer und Operationen implementiert werden, die Rahmenbedingungen hierfür sind aber für eine zukünftige Erweiterung der Anwendung gegeben und funktionsfähig.
\\Die Implementierung bis zu diesem Punkt war umfangreich, die Aufgaben der Lehrerseite umfassten aber leider kaum neue Komponenten. Dennoch wurde erstmals mit XIB-Dateien gearbeitet, UI und Contraints im Code erzeugt und mit Operations gearbeitet. Somit konnten trotzdem neue Techniken und Funktionalitäten getestet und erlernt werden.
\chapter{Gemeinsamen Datenmodell zur Speicherung der Schnittstellendaten}
Christian Pfeiffer \& Bastian Kusserow
\section{Ziele}
Zur Umsetzung des temporären Datenmodells haben wir uns folgende Ziele gesetzt:

\begin{itemize}
\item Vermeidung von Redundanz
\item Wiederverwendbarkeit von Code
\item Ressourcenschonendes Speichern von heruntergeladenen Schnittstellendaten
\item Vereinheitlichter Zugriff und Downloadabwicklung (Fetch) von Schnittstellendaten
\end{itemize}

\section{Implementierung}

\subsection{TKModelSingleton} \label{TKModelSingleton}
Um das Ziel des Vereinheitlichten Zugriffs auf Schnittstellendaten zu erfüllen, wird eine Singleton Klasse zum Abspeichern der Schnittstellendaten verwendet.
Das Besondere an einem Singleton ist, dass es immer nur ein Objekt der Klasse existiert, welches von der Singleton Klasse selbst erzeugt und verwaltet wird. 

\lstinputlisting[language=swift]{source/fetchsingleton.swift}

In unseren \texttt{TKModelSingleton} werden verschiedene Variablen gehalten, welche an verschiedenen anderen Stellen in der App benötigt werden. Die Konstante \texttt{sharedInstance} hält die Instanz des Singleton. Die Variable \texttt{downloadedClasses} hält die heruntergeladenen TKClass Objekte in einem Array, welche von der private Database des Nutzers gefetched wurden. In der Variable \texttt{downloadedSubjects} werden die TKSubjects abgespeichert, welche von der shared Database des Nutzers stammen. Die Variable \texttt{myTKRank} enthälgt den TKRank auf welchen der Controller bei dem letzten Fetch initialisiert wurde.


\subsection{TKFetchController}
Als Zugriffsschicht auf den TKModelSingleton wurde der TKFetchController implementiert. In ihm sind einerseits Getter und Setter Methoden für den Zugriff auf den Singleton implementiert. Andererseits ist in diesem auch die Logik für die Fetch Funktionalitäten von der Schnittstelle implementiert.

Im folgenden Codeblock soll exemplarisch dargestellt werden, wie Daten aus dem TKFetchSingleton bereit gestellt werden:
\lstinputlisting[language=swift]{source/getSubjAndDoc.swift}
Um die Daten aus dem Singleton abzurufen, wurden verschiedenartige Getter und Setter Methodiken implementiert. Exemplarisch soll hier eine etwas komplexere Get-Methode dargestellt werden:
Die Methode: \texttt{getSubjectAndDocumentForCollectionIndex()} liefert Beispielsweise für die \texttt{CardViews} im Schülerhauptmenü das benötigte \texttt{TKSubject} \& \texttt{TKDocument} für die View.

\subsubsection{Abrufen aus der Schnittstelle mit Operations}
Da die Schnittstelle die Daten mit vier verschiedenen verschachtelten Datentypen bereitstellt, soll im Folgenden dargestellt werden, wie die Schnittstellendaten mithilfe von Operations abgerufen werden können.
\lstinputlisting[language=swift]{source/fetchAll.swift}

\textbf{1: Zuruecksetzen der des Controllers und initialisieren der Operations}
Bevor die Operations initialisiert werden können, werden mit der Methode \texttt{resetWithRank()} die in dem \texttt{TKModelSingleton} gehaltenen Variablen zurückgesetzt und der \texttt{TKRank} mit den neuen \texttt{TKRank} aktualisiert. Danach werden alle Operations mit den neuen \texttt{TKRank} initialisert.

\textbf{2: Definieren der Completion Blocks nach der Operations}
Anschließend werden die Completion Blocks der Operations definiert. Hierbei werden die jeweiligen heruntergeladenen Objekte von der Schnittstelle in die darauffolgende Operation eingefügt. In dem \texttt{Completion Block} der Exercise Operation wird eine Notification an die \texttt{UIThreads} geschickt, welche anschließend die neuen von der Schnittstelle abgerufenen Daten abrufen können.

\textbf{3: Setzen der Operation Dependencies und Aufsetzen der Queue}
Hier werden die Dependencies (Abhängikeiten) der Operations zueinander gesetzt. Es muss immer die darüberliegende Operation den \texttt{Completion Block} abgehandelt haben, bevor die darunterliegende Operation beginnen kann. Anschließend werden die Operations mithilfe einer Queue in die richtige Reihenfolge gebracht.

Weiter Informationen über die Funktionsweise von den Operations ist in dem Leherteil dieser Dokumentation enthalten. %todo: Leherteil verlinken sobald der existiert info an lehrerteil mit \label...
\chapter{Schnittstelle iCloud}
Patrick Niepel, Marcel Hagmann, Carl Philipp Knoblauch

\section{Einleitung}
In diesem Abschnitt wird die Schnittstelle mit iCloud beschrieben. Dabei wird erklärt wie die Architektur aufgebaut ist, wie mit der Schnittstelle kommuniziert wird und welche Probleme aufgetreten sind. 

\section{Warum iCloud/CloudKit?}

Wenn man bei der Entwicklung einer iOS App auf Cloud Services zurückgreifen will, bietet sich natürlich das Apple eigene CloudKit für iCloud an. Es ergab sich dadurch auch die Möglichkeit eine neues Framework kennenzulernen, da wir zuvor noch nicht mit CloudKit gearbeitet hatten. 
Über die Cloud-Schnittstelle sollen zwischen Lehrer und Schüler alle Aufgaben geteilt werden. Der Lehrer kann seine Aufgaben/Spiele in iCloud laden und diese mit seinen Schülern teilen. Die Schüler sollen dann diese Aufgaben/Spiele erledigen und ihre Lösungen wieder in iCloud laden. Dadurch wird dem Lehrer wiederum ermöglicht sich einen Überblick über die Lösungen seiner Klasse zu machen. Mit CloudKit konnten wir diese Ziele alle umsetzten.

\section{Architektur}


\begin{figure}
  \includegraphics[width=\linewidth]{images/Klassendiagram_TeachKit.jpg}
  \caption{TeachKit Schnittstellen Klassendiagramm}
  \label{fig:diagramTeachKit}
\end{figure}

Auch TeachKit ist strikt nach der Model-View-Controller - Architektur aufgebaut. Hierbei erben alle Models, die in der Cloud gespeichert werden, von ihrer Superklasse TKCloudObject. Die Controller die diese Models verwalten, sind mit Hilfe eines generischen Controllers TKGenericCloudController implementiert worden. Alle Cloud-Models und Controller bedienen sich von verschiedenen Enumerations, die die Funktionalität übersichtlicher gestalten.

\newpage


\section{Features}

\subsection{Upload/Download/Delete/Fetch/Update}

Für jeden Datentyp in TeachKit, gibt es einen Controller der für die Operationen auf den Datentyp verantwortlich ist.
Somit gibt es folgende Controller für die Datentypen:

\begin{itemize}

\item TKClassController
\item TKSubjectController
\item TKDocumentController
\item TKExerciseController
\item TKSolutionController

\end{itemize}

Alle Controller sind ähnlich aufgebaut. Der Grund weshalb der Controller über die Methode initialize(…) funktionsfähig gemacht werden muss ist der, dass der Student auf die Shared-Database zugreift und diese zuerst gefetched werden muss.

Um doppelten Code zu vermeiden, arbeiten alle der oben genannten Controller mit dem TKGenericCloudController, der die Grundfunktionen übernimmt und in den jeweiligen Controllern dann spezialisiert werden.

\subsection{User Profile}

Zu jedem Nutzer kann der Vorname, Nachname und ein Profilbild gespeichert werden. Der TKUserProfileController der für die Nutzerverwaltung verantwortlich ist, arbeitet auf der Public-Database. Das bedeutet, dass alle Nutzer diese Informationen sehen können. Die aktuelle implementation erlaubt nur den download der Daten für den derzeit eingeloggten Nutzer. Diese kann erweitert werden, hätte aber momentan keine Verwendung gefunden.

\newpage

\subsection{Sharing}

Eines der wichtigsten Features unserer App ist das Teilen von Daten zwischen Teacher und Student. Nach dem Teilen gemeinsamer Daten haben beide Zugriff auf das Subject und alles was darunter angelegt ist.

CloudKit arbeitet mit drei verschiedenen Datenbanken.


\textbf{Private-Database}
\newline
Der aktuell angemeldete Nutzer ist der Inhaber der Daten, nur dieser hat Zugriff auf diese Datenbank und hat das Recht zu lesen und zu schreiben.
\newline
\textbf{Shared-Database}
\newline
Der aktuelle Nutzer ist nicht der Besitzer der geteilten Daten, und hat die beim Teilen zugewiesenen lese und/oder schreib Rechte.
\newline
\textbf{Public-Database}
\newline
Der aktuelle Nutzer ist nicht der Besitzer der geteilten Daten, und hat die beim Teilen Jeder App Nutzer hat lese Recht auf diese Daten, auch ohne aktiven iCloud Account.
\newline

Legt der Teacher seine Daten an, befinden diese sich in seiner private-Database. Nachdem dieser das Subject geteilt hat, befindet sich dies immer noch in der private-Database. Bei jedem Student der nun berechtigt ist, das geteilte Subject einzusehen, wird eine Referenz in der shared-Database gespeichert. Diese Referenz zeigt auf die Daten des Teachers in der private-Database.

Der Teacher teilt mit dem Student ein Subject, damit der Student die neue Arbeitsblätter einsehen kann. Mit dem Student wird der Zugriff auf Class nicht geteilt, weil er diese Information nicht benötigt.

\begin{center}
   \includegraphics[width=5cm]{sharedDB_privateDB.pdf}
   \captionof{figure}{hared DB / Private DB}
\end{center}

Das Teilen findet über den TKShareController statt. Mit der Methode createCloudSharingController(…), wird der ViewController der für das Teilen verantwortlich ist erstellt und kann angezeigt werden.
Bei der Verwendung des Controllers müssen keine weiteren Bedingungen beachtet werden, der Controller kümmert sich um alles was zum Teilen benötigt wird.

\subsection{Push/Subscriptions}

Eine Besonderheit des CloudKits ist die einfache Implementierung von Push Benachrichtigungen bei Datenbankänderungen. PushNotifications in einer App zu implementieren, die über einen eigenen Server laufen, setzen PHP Kenntnisse voraus und können einiges an Zeit in Anspruch nehmen.
CloudKit Subscriptions sind jedoch einfacher zu implementieren und können auf Insert, Update, Delete und Create in der Datenbank reagieren.

Der Verwendungszweck der Subscriptions war dafür gedacht, dass der Student über Änderungen zu seinem Fach informiert wird. Dies könnte beispielsweise der Fall sein, wenn der Professor ein neues Arbeitsblatt seinem Fach hinzufügt.

Wir implementierten die CloudKit Subscriptions, die darauf reagiert, wenn der Teacher ein neues Document einem Subject hinzufügt.
Die ersten Tests auf zwei unterschiedlichen Geräten mit der gleichen Apple ID waren erfolgreich. Beim Testen über zwei unterschiedliche Apple ID’s, bei dem der Teacher ein Subject mit dem Student teilt, kamen wir allerdings an unsere Grenzen. Der Student wurde beim Hinzufügen eines neuen Objekts nicht benachrichtigt.
In der CloudKit API wurden wir dann auf die folgende Anmerkung aufmerksam, die besagt, dass Subscriptions auf der Shared-Database nicht unterstützt werden.

\newpage

\subsection{TKError}

Bei den meisten Aktionen die innerhalb des TeachKits Fehler auslösen können, werden Fehler vom Datentyp TKError erstellt. Jeder Fehler beinhaltet eine genauere Beschreibung des aufgetretene Problems. Folgende Fehler existieren:

\textbf{networkUnavailable}
Es besteht keine Internetverbindung.

\textbf{networkFailure}
Es besteht eine Internetverbindung, es konnte allerdings keine Verbindung zur Cloud hergestellt werden.

\textbf{wrongRecordZone}
Die Operation wird auf der falschen RecordZone ausgeführt oder existiert nicht. An Schnittstelle wenden.

\textbf{failedSharing}
Das Objekt konnte nicht geteilt werden.

\textbf{parentObjectIsFaulty}
Operation konnte nicht ausgeführt werden, da das Parent-Objekt Fehlerhaft ist.

\textbf{objectIsFaulty}
Operation konnte nicht ausgeführt werden, da das Objekt Fehlerhaft ist.

\textbf{objectIsFaultyAfterCloudUpload}
Auf dem Objekt wurde eine Operation in der Cloud ausgeführt. Das von der Cloud erhaltene Objekt ist inkonsistent.

\textbf{userCouldNotLoad}
Beim Zugriff auf die Nutzer Informationen ist ein Fehler unterlaufen.

\textbf{updateOperationFailed}
Der Datensatz konnte nicht geupdated werden.

\textbf{deleteOperationFailed}
Der Datensatz konnte nicht gelöscht werden.

\textbf{createOperationFailed}
Der Datensatz konnte nicht erstellt werden.

\textbf{addOperationFailed}
Der Datensatz konnte nicht hinzugefügt werden.

\textbf{fetchSortTypeNotAvailable}
Das Attribut nach dem sortiert werden soll existiert nicht.

\textbf{noWritePermission}
Der Nutzer hat keine Berechtigung die Daten zu ändern.

\textbf{noSharedData}
Die Operation konnte nicht ausgeführt werden, da noch keine Daten geteilt werden.

\newpage

\section{Aufgetretene Probleme}

\subsection{Subscription}
Da auf das Problem mit den Subscriptions bereits im Abschnitt Push/Subscriptions eingegangen wurde, erwähnen wir an dieser Stelle nur noch einmal, das Subscriptions nicht auf der Shared-Database unterstützt werden.


\subsection{Sharing}
Während der Implementation der Sharing Funktion, hatten wir über einen längeren Zeitraum das Problem, dass das Record zwischen den Nutzern geteilt wurde. Der darunter hängende Baum war allerdings nicht beim Student einzusehen.

\begin{center}
   \includegraphics[width=5cm]{images/problemImage.pdf}
   \captionof{figure}{Zugriff Teacher/Student}
\end{center}

Letztendlich stellte sich heraus, dass die Verbindung zwischen Parent- und dem Child-Record nicht hergestellt wurde. Dieses Problem konnte ganz einfach mit der Methode setParent(CKRecord?) gelöst werden, allerdings musste man dafür erst wissen, dass so eine Funktion überhaupt existiert.

\newpage

\subsection{TKSolution}

Die Idee hinter TKSolution war die, dass ein Student seine Lösung darin erstellt und anschließend zu der TKExercise hinzufügt. Allerdings ist es dem Student nicht möglich, weitere Records in der geteilten Datenbank des Teachers zu erstellen und hinzuzufügen.

\begin{center}
   \includegraphics[width=15cm]{images/code_error.pdf}
   \captionof{figure}{Error}
\end{center}

Da diese Fehlermeldung auftritt obwohl wir die publicPermission des CKShare Objektes richtig gesetzt haben und immer noch kein Record erzeugen konnten, mussten wir eine andere Lösung dafür suchen. Wir entschlossen uns die Lösungen in ein Data-Objekt zu serialisieren und diese in TKExercise hinzuzufügen. 
Unseren neuen Lösungsansatz testen wir zuerst mit einem Attribut vom Datentyp String und nicht den eigentlich benötigten Datentyp Data. Beim ersten Upload in die iCloud werden die Records in der Cloud automatisch angelegt. Der erste Test hat geklappt und anschließend wollten wir unseren benötigten Daten vom Typ Data sichern. Da allerdings die angelegten Constraints nicht mehr für den neuen Datentyp gestimmt haben, konnten keine Records mehr hochgeladen werden. Die Constraints müssen zuerst im iCloud-Dashboard gelöscht und erneut hochgeladen werden.



%usw. ...


\listoffigures
\end{document}
